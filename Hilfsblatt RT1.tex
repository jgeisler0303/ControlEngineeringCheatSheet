%% LyX 2.4.0~RC3 created this file.  For more info, see https://www.lyx.org/.
%% Do not edit unless you really know what you are doing.
\documentclass[ngerman]{scrartcl}
\usepackage[T1]{fontenc}
\usepackage[utf8]{inputenc}
\usepackage[landscape,a4paper]{geometry}
\geometry{verbose,tmargin=2cm,bmargin=1cm,lmargin=1cm,rmargin=1cm}
\usepackage{array}
\usepackage{varwidth}
\usepackage{xltabular}
\usepackage{amsmath}
\usepackage{graphicx}

\makeatletter

%%%%%%%%%%%%%%%%%%%%%%%%%%%%%% LyX specific LaTeX commands.
%% Because html converters don't know tabularnewline
\providecommand{\tabularnewline}{\\}
%% Variable width box for table cells
\newenvironment{cellvarwidth}[1][t]
    {\begin{varwidth}[#1]{\linewidth}}
    {\@finalstrut\@arstrutbox\end{varwidth}}

%%%%%%%%%%%%%%%%%%%%%%%%%%%%%% User specified LaTeX commands.
\usepackage{cellspace}
\setlength\cellspacetoplimit{5pt}
\setlength\cellspacebottomlimit{5pt}

\makeatother

\usepackage{babel}
\begin{document}
\begin{center}
{\Large\textbf{Offizielles Hilfsblatt für die Klausur Regelungstechnik
1, Seite 1/2}}{\Large\par}
\par\end{center}

\begin{center}
Prof. Dr.-Ing. Jens Geisler, Hochschule Flensburg, Stand: \today
\par\end{center}

\begin{xltabular}[c]{2cm}{|c|c|c|c|c|c|c|c|}
\hline 
 & Differenzialgleichung
\begin{varwidth}{\linewidth}
\rule{0pt}{1.4cm}
\end{varwidth}
 & \begin{cellvarwidth}[m]
\centering
Übertragungsfunktion\\
$G(s)$
\end{cellvarwidth} & \begin{cellvarwidth}[m]
\centering
Übergangsfunktion

$h(t)$
\end{cellvarwidth} & \begin{cellvarwidth}[m]
\centering
Ortskurve\\
$G(\textrm{j}\omega)$
\end{cellvarwidth} & \begin{cellvarwidth}[m]
\centering
Bode-\\
Amplitudengang
\end{cellvarwidth} & \begin{cellvarwidth}[m]
\centering
Bode-\\
Phasengang
\end{cellvarwidth} & \begin{cellvarwidth}[m]
\centering
Pol-Nullstellen-\\
Diagramm
\end{cellvarwidth}\tabularnewline
\hline 
\endhead
\hline 
PT1 & $T\dot{x}_{a}(t)+x_{a}(t)=K\cdot x_{e}(t)$ & $\frac{K}{Ts+1}$
\begin{varwidth}{\linewidth}
\rule{0pt}{2.2cm}
\end{varwidth}
 & 
\begin{varwidth}{\linewidth}
\includegraphics{IconStep_PT1}
\end{varwidth}
 & 
\begin{varwidth}{\linewidth}
\includegraphics{IconNyquist_PT1}
\end{varwidth}
 & 
\begin{varwidth}{\linewidth}
\includegraphics{IconBodeA_PT1}
\end{varwidth}
 & 
\begin{varwidth}{\linewidth}
\includegraphics{IconBodeP_PT1}
\end{varwidth}
 & 
\begin{varwidth}{\linewidth}
\includegraphics{IconPoleZero_PT1}
\end{varwidth}
\tabularnewline
\hline 
PT2 & \begin{cellvarwidth}[m]
\centering
~$\begin{gathered}T_{1}T_{2}\ddot{x}_{a}(t)+(T_{1}+T_{2})\dot{x}_{a}(t)+x_{a}(t)\\
=K\cdot x_{e}(t)
\end{gathered}
$\\
{\footnotesize (aus 2 PT1-Gliedern)}
\end{cellvarwidth} & $\frac{K}{\left(T_{1}s+1\right)\left(T_{2}s+1\right)}$
\begin{varwidth}{\linewidth}
\rule{0pt}{2.2cm}
\end{varwidth}
 & 
\begin{varwidth}{\linewidth}
\includegraphics{IconStep_PT1x2}
\end{varwidth}
 & 
\begin{varwidth}{\linewidth}
\includegraphics{IconNyquist_PT1x2}
\end{varwidth}
 & 
\begin{varwidth}{\linewidth}
\includegraphics{IconBodeA_PT1x2}
\end{varwidth}
 & 
\begin{varwidth}{\linewidth}
\includegraphics{IconBodeP_PT1x2}
\end{varwidth}
 & 
\begin{varwidth}{\linewidth}
\includegraphics{IconPoleZero_PT1x2}
\end{varwidth}
\tabularnewline
\hline 
PT2 & \begin{cellvarwidth}[m]
\centering
$\begin{gathered}\omega_{0}^{-2}\ddot{x}_{a}(t)+2D\omega_{0}^{-1}\dot{x}_{a}(t)+x_{a}(t)\\
=K\cdot x_{e}(t)
\end{gathered}
$\\
{\footnotesize (schwingfähig)}
\end{cellvarwidth} & $\frac{K}{\frac{1}{\omega_{0}^{2}}s^{2}+\frac{2D}{\omega_{0}}s+1}$
\begin{varwidth}{\linewidth}
\rule{0pt}{2.2cm}
\end{varwidth}
 & 
\begin{varwidth}{\linewidth}
\includegraphics{IconStep_PT2}
\end{varwidth}
 & 
\begin{varwidth}{\linewidth}
\includegraphics{IconNyquist_PT2}
\end{varwidth}
 & 
\begin{varwidth}{\linewidth}
\includegraphics{IconBodeA_PT2}
\end{varwidth}
 & 
\begin{varwidth}{\linewidth}
\includegraphics{IconBodeP_PT2}
\end{varwidth}
 & 
\begin{varwidth}{\linewidth}
\includegraphics{IconPoleZero_PT2}
\end{varwidth}
\tabularnewline
\hline 
IT1 & $T\ddot{x}_{a}(t)+\dot{x}_{a}(t)=K\cdot x_{e}(t)$ & $\frac{K}{s\left(Ts+1\right)}$
\begin{varwidth}{\linewidth}
\rule{0pt}{2.2cm}
\end{varwidth}
 & 
\begin{varwidth}{\linewidth}
\includegraphics{IconStep_IT1}
\end{varwidth}
 & 
\begin{varwidth}{\linewidth}
\includegraphics{IconNyquist_IT1}
\end{varwidth}
 & 
\begin{varwidth}{\linewidth}
\includegraphics{IconBodeA_IT1}
\end{varwidth}
 & 
\begin{varwidth}{\linewidth}
\includegraphics{IconBodeP_IT1}
\end{varwidth}
 & 
\begin{varwidth}{\linewidth}
\includegraphics{IconPoleZero_IT1}
\end{varwidth}
\tabularnewline
\hline 
DT1 & $T\dot{x}_{a}(t)+x_{a}(t)=K\cdot\dot{x}_{e}(t)$ & $\frac{K\cdot s}{Ts+1}$
\begin{varwidth}{\linewidth}
\rule{0pt}{2.2cm}
\end{varwidth}
 & 
\begin{varwidth}{\linewidth}
\includegraphics{IconStep_DT1}
\end{varwidth}
 & 
\begin{varwidth}{\linewidth}
\includegraphics{IconNyquist_DT1}
\end{varwidth}
 & 
\begin{varwidth}{\linewidth}
\includegraphics{IconBodeA_DT1}
\end{varwidth}
 & 
\begin{varwidth}{\linewidth}
\includegraphics{IconBodeP_DT1}
\end{varwidth}
 & 
\begin{varwidth}{\linewidth}
\includegraphics{IconPoleZero_DT1}
\end{varwidth}
\tabularnewline
\hline 
\end{xltabular}

\pagebreak{}
\begin{center}
{\Large\textbf{Offizielles Hilfsblatt für die Klausur Regelungstechnik
1, Seite 2/2}}{\Large\par}
\par\end{center}

\begin{center}
Prof. Dr.-Ing. Jens Geisler, Hochschule Flensburg, Stand: \today
\par\end{center}

\begin{xltabular}[c]{2cm}{|c|c|c|c|c|c|c|c|}
\hline 
 & Differenzialgleichung
\begin{varwidth}{\linewidth}
\rule{0pt}{1.4cm}
\end{varwidth}
 & \begin{cellvarwidth}[m]
\centering
Übertragungsfunktion\\
$G(s)$
\end{cellvarwidth} & \begin{cellvarwidth}[m]
\centering
Übergangsfunktion

$h(t)$
\end{cellvarwidth} & \begin{cellvarwidth}[m]
\centering
Ortskurve\\
$G(\textrm{j}\omega)$
\end{cellvarwidth} & \begin{cellvarwidth}[m]
\centering
Bode-\\
Amplitudengang
\end{cellvarwidth} & \begin{cellvarwidth}[m]
\centering
Bode-\\
Phasengang
\end{cellvarwidth} & \begin{cellvarwidth}[m]
\centering
Pol-Nullstellen-\\
Diagramm
\end{cellvarwidth}\tabularnewline
\hline 
\endhead
\hline 
PI & $y(t)=K_{R}\left(e(t)+\frac{1}{T_{N}}\int e(t)\thinspace\textrm{d}\tau\right)$ & $K_{R}\frac{T_{N}s+1}{T_{N}s}$
\begin{varwidth}{\linewidth}
\rule{0pt}{2.2cm}
\end{varwidth}
 & 
\begin{varwidth}{\linewidth}
\includegraphics{IconStep_PI}
\end{varwidth}
 & 
\begin{varwidth}{\linewidth}
\includegraphics{IconNyquist_PI}
\end{varwidth}
 & 
\begin{varwidth}{\linewidth}
\includegraphics{IconBodeA_PI}
\end{varwidth}
 & 
\begin{varwidth}{\linewidth}
\includegraphics{IconBodeP_PI}
\end{varwidth}
 & 
\begin{varwidth}{\linewidth}
\includegraphics{IconPoleZero_PI}
\end{varwidth}
\tabularnewline
\hline 
PD & $x_{a}(t)=K_{R}\left(e(t)+T_{V}\frac{\textrm{d}}{\textrm{d}t}e(t)\right)$ & $K_{R}\left(T_{V}s+1\right)$
\begin{varwidth}{\linewidth}
\rule{0pt}{2.2cm}
\end{varwidth}
 & 
\begin{varwidth}{\linewidth}
\includegraphics{IconStep_PD}
\end{varwidth}
 & 
\begin{varwidth}{\linewidth}
\includegraphics{IconNyquist_PD}
\end{varwidth}
 & 
\begin{varwidth}{\linewidth}
\includegraphics{IconBodeA_PD}
\end{varwidth}
 & 
\begin{varwidth}{\linewidth}
\includegraphics{IconBodeP_PD}
\end{varwidth}
 & 
\begin{varwidth}{\linewidth}
\includegraphics{IconPoleZero_PD}
\end{varwidth}
\tabularnewline
\hline 
PDT1 & $\begin{gathered}T_{p}\dot{x}_{a}(t)+x_{a}(t)\\
=K_{R}\left(e(t)+T_{V}\frac{\textrm{d}}{\textrm{d}t}e(t)\right)
\end{gathered}
$ & $K_{R}\frac{T_{V}s+1}{T_{p}s+1}$
\begin{varwidth}{\linewidth}
\rule{0pt}{2.2cm}
\end{varwidth}
 & 
\begin{varwidth}{\linewidth}
\includegraphics{IconStep_PDT1}
\end{varwidth}
 & 
\begin{varwidth}{\linewidth}
\includegraphics{IconNyquist_PDT1}
\end{varwidth}
 & 
\begin{varwidth}{\linewidth}
\includegraphics{IconBodeA_PDT1}
\end{varwidth}
 & 
\begin{varwidth}{\linewidth}
\includegraphics{IconBodeP_PDT1}
\end{varwidth}
 & 
\begin{varwidth}{\linewidth}
\includegraphics{IconPoleZero_PDT1}
\end{varwidth}
\tabularnewline
\hline 
P & 
\begin{varwidth}{\linewidth}
\rule{0pt}{2.2cm}
\end{varwidth}
 &  &  &  &  &  & \tabularnewline
\hline 
I & 
\begin{varwidth}{\linewidth}
\rule{0pt}{2.2cm}
\end{varwidth}
 &  &  &  &  &  & \tabularnewline
\hline 
D & 
\begin{varwidth}{\linewidth}
\rule{0pt}{2.2cm}
\end{varwidth}
 &  &  &  &  &  & \tabularnewline
\hline 
\end{xltabular}
\end{document}
